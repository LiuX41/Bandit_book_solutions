%!TEX root =  main.tex

\chapter*{Chapter 3 Stochastic Processes and Markov Chains}
\label{sec:second}

\noindent\textbf{3.1} Fill in the details of Theorem 3.1:


\begin{enumerate}
   \item[(a)] Prove that $F_t \in \set{0,1}$ is a Bernoulli random variable for all $t \ge 1$.
   \item[(b)] In what follows, equip $\cS$ with $\mathbb{P}=\lambda$, the uniform probability measure. Show that for any $t \ge 1$, $F_t$ is uniformly distributed: $\PP{F_t= 1 } = \PP{F_t = 0} = 1/2$. 
   \item[(c)] Show that $(F_t)_{t=1}^{\infty}$ are independent.
   \item[(d)] Show that $(X_{m,t})_{t=1}^\infty$ is an independent sequence of Bernoulli random variables that are uniformly distributed.
   \item[(e)] Show that $X_t = \sum_{t=1}^\infty X_{m,t}2^{-t}$ is uniformly distributed on $[0,1]$
   \item[(f)] Show that $(X_t)_{t=1}^\infty$ are independent.
\end{enumerate}

\begin{proof}
    \begin{enumerate}
   \item[(a)] It can be shown that $F_t$ has the following form: 
   $$F_t(x) = \bOne{x\in U_t}, U_t = \set{1} \cup \bigcup_{0 \le s\le 2^{t-1}}\left[\frac{2s-1}{2^t}, \frac{2s}{2^t}\right)$$. \\
   Since $U_t \in \cB([0,1])$, thus we can show $F_t$ is $\cB([0,1])$-mearuable. And $F_t \in \set{0,1}$, thus it is Bernoulli random variable. 
   \item[(b)] $\PP{F_t = 1} = \PP{U_t} = \sum_{s=0}^{2^{t-1}}\frac{1}{2^t} =\frac{2^{t-1}}{2^t} =\frac{1}{2}$. And $\PP{F_t = 0} = 1-\PP{F_t = 1}=\frac{1}{2}$.
   \item[(c)] Given an index set $K \subseteq \NN+$ we need to show that $\set{F_k,k\in K}$ are independent. Or equivalently, that
   \begin{align*}
      \PP{\bigcap_{k \in K}} = \prod_{k \in K} \PP{U_k} = 2^{-|K|}\,.
   \end{align*}
   Let $k=\max K$. Then 
   \begin{align*}
      \lambda \bracket{U_k \cap \cup_{j \in K \setminus \set{k}} U_j } = \frac{1}{2} \lambda \bracket{\cup_{j \in K \setminus \set{k}} U_j}\,.
   \end{align*}
   Then by induction we can get the desired result. 
   \item[(d)] It follows directly from the definition of independence that any subsequence of an independent sequence is also an independent sequence. Thus from (b) we can get
   \begin{align*}
      \PP{X_{m,t} = 0 } = \PP{X_{m,t} = 1 } =\frac{1}{2}\,.
   \end{align*}
   \item[(e)] 
By definition, $X_t = \sum_{t=1}^{\infty}$ is a weighted sum of an independent sequence of uniform Bernoully random variables. Thus $X_t$ has the same property as random variable $Y = \sum_{t=1}^\infty F_t 2^{-t}$, and $Y(x) = x, \PP{Y \le x} = x$ thus we can also conclude $X_t$ is uniformly distributed. 

   \item[(f)] The disjoint subsets of independent random variables are also independent. 
\end{enumerate}
\end{proof}




% \begin{enumerate}[(a)]
%     \item On$([0,1],\mathcal{B},\lambda)$,for any $x\in[0,1]$
    
%     Let $F_1(x),F_2(x),F_3(x)$,...be the binary expansion of x.
    
%     $F_t(x)$=
%     $\begin{cases}
%     1 , A\\
%     0 , \overline{A}\quad(\overline{A}\text{ is the opposite case of }A)
%     \end{cases}$
    
%     $F_t(x)$ is Bernoulli random variable.
    
%     \item $\begin{cases}
%     F_1=0:0\le x<0.5\\
%     F_1=1:0.5\le x<1
%     \end{cases}$
    
%     $\begin{cases}
%     F_2=0:0\le x'<0.5\\
%     F_2=1:0.5\le x'<1
%     \end{cases}$
%     \label (x'=2x-1
    
%     ...
    
%     $\begin{cases}
%     F_t=0:0\le x^t<0.5 \Rightarrow \mathbb{P}(F_t=0)=\frac{1}{2}\\
%     F_t=1:0.5\le x^t<1 \Rightarrow \mathbb{P}(F_t=1)=\frac{1}{2}
%     \end{cases}$
    
%     \item It is obviously that $(F_t)^\infty_{t=1}$ are independent. It satisfies independent equation: $\mathbb{P}(A\cap B)=\mathbb{P}(A)\mathbb{P}(B)$.
    
%     \item $(X_{m,t})^\infty_{t=1}$ is a subsequence of $(F_t)^\infty_{t=1}$ and $(X_{m,t})^\infty_{t=1}$ are mutually exclusive.
%     \item Such as(d).
%     \item Such as(d).
%     \end{enumerate}

\noindent\textbf{3.2}
(\textsc{Martingales and optional stopping}) Let $\left(X_{t}\right)_{t=1}^{\infty}$ be an infinite sequence of independent
Rademacher random variables and $S_{t}=\sum_{s=1}^{t} X_{s} 2^{s-1}$.
\begin{enumerate}
    \item[(a)] Show that $\left(S_{t}\right)_{t=0}^{\infty}$ is a martingale.
    \item[(b)] Let $\tau=\min \left\{t: S_{t}=1\right\}$ and show that $\mathbb{P}(\tau<\infty)=1$. 
    \item[(c)] What is $\mathbb{E}\left[S_{\tau}\right]$?
    \item[(d)] Explain why this does not contradict Doob's optional stopping theorem. 
\end{enumerate}

\begin{proof}
    \begin{enumerate}
        \item[(a)] First of all, we can observe that $S_t$ is $\mathcal{F}_{t}$-measurable given that it is actually a function of random variables $X_1, \cdots, X_t$ on $(\Omega, \mathcal{F}, \mathbb{P})$.
        
        Next, we have
        \begin{equation*}
            \begin{aligned}
                \mathbb{E}\left[S_{t} \mid \mathcal{F}_{t-1}\right] 
                &= \mathbb{E}\left[S_{t-1} + X_t 2^{t-1}\mid \mathcal{F}_{t-1}\right]\\
                &= \mathbb{E}\left[S_{t-1} \mid \mathcal{F}_{t-1}\right] + \mathbb{E}\left[X_t 2^{t-1} \mid \mathcal{F}_{t-1}\right]\\
                &= S_{t-1} + \mathbb{E}\left[X_t 2^{t-1} \mid \mathcal{F}_{t-1}\right]\\
                &= S_{t-1} + \frac{1}{2} 2^{t-1} + \frac{1}{2} (-2^{t-1})\\
                &= S_{t-1}.
            \end{aligned}
        \end{equation*}

        Finally, as a weighted sum of finite random varialbes, $S_t$ is surely integrable.

        \item[(b)] Notice that 
        \begin{equation*}
            \begin{aligned}
                \mathbb{P}(\tau = n) 
                &= \mathbb{P}(\tau \neq 1) \mathbb{P}(\tau \neq 2 \mid \tau \neq 1) \cdots \mathbb{P}(\tau = n \mid \tau \notin \{1, \cdots n - 1\})\\
                &= \mathbb{P}(X_1 = -1) \mathbb{P}(X_2 = -1 \mid X_1 = -1) \cdots \mathbb{P}(X_n = 1 \mid X_1 = \cdots = X_{n-1} = -1)\\
                &= \frac{1}{2^n},
            \end{aligned}
        \end{equation*}
        which implies $\mathbb{P}(\tau = \infty) = \lim_{n \rightarrow \infty} \mathbb{P}(\tau = n) = 0$ and hence $\mathbb{P}(\tau < \infty) = 1 - \mathbb{P}(\tau = \infty) = 1$.

        \item[(c)] It is obvious that $\mathbb{E}(S_\tau) = 1$ according to our stopping rule.
        \item[(d)] The result of (c) does not contradict Doob's optional stopping theorem because the conditions are not satisfied.
        Firstly, $\mathbb{P}(\tau > n) = \frac{1}{2^n} > 0$ for any $n \in \mathbb{N}$, which breaks the first condition.
        Next, $|S_{t+1} - S_{t}| = 2^t$ and $|S_{t \wedge \tau}| = |\sum_{s=1}^{t \wedge \tau} X_s 2^{s-1}$ can neither not be bounded, which breaks the second and third condition.
    \end{enumerate}
\end{proof}

\noindent\textbf{3.3}
(\textsc{Martingales and optional stopping (ii)}) Give an example of a martingale $\left(S_{n}\right)_{n=0}^{\infty}$ and
stopping time $\tau$ such that $$\lim _{n \rightarrow \infty} \mathbb{E}\left[S_{\tau \wedge n}\right] \neq \mathbb{E}\left[S_{\tau}\right].$$
\begin{proof}
    Notice that $(S_{\tau \wedge n})_n^\infty$ is also a martingale (which is usually called stopped martingale).
    This can be shown by formulating $S_{\tau \wedge n} = S_{\tau \wedge (n-1)} + \mathbb{I}\left\{\tau \leq n\right\} (S_n - S_{n-1})$ and checking
    \begin{equation*}
        \begin{aligned}
            \mathbb{E}[S_{\tau \wedge n} | \mathcal{F}_{n-1}]
            &= \mathbb{E}[S_{\tau \wedge (n-1)} + \mathbb{I}\left\{\tau \leq n\right\} (S_n - S_{n-1}) | \mathcal{F}_{n-1}]\\
            &= \mathbb{E}[S_{\tau \wedge (n-1)} | \mathcal{F}_{n-1}] + \mathbb{E}[\mathbb{I}\left\{\tau \leq n\right\} (S_n - S_{n-1}) | \mathcal{F}_{n-1}]\\
            &= S_{\tau \wedge (n-1)} + \mathbb{I}\left\{\tau \leq n\right\}(S_{n-1} - S_{n-1})\\
            &= S_{\tau \wedge (n-1)}.
        \end{aligned}
    \end{equation*}

    Therefore, we have $\mathbb{E}(S_{\tau \wedge n}) = \mathbb{E}(S_{\tau \wedge 0}) = \mathbb{E}(S_{0})$.
    Now it suffices to give the same example as in 3.2 that $\lim _{n \rightarrow \infty} \mathbb{E}\left[S_{\tau \wedge n}\right] = \mathbb{E}(S_{0}) \neq \mathbb{E}\left[S_{\tau}\right]$.
\end{proof}


\noindent\textbf{3.4} If $X_t\ge0$ is dropped, $\mathbb{E}[X_\tau|\{\tau\le n\}] \ge \mathbb{E}[\varepsilon|\{\tau\le n\}]$ not always true.




\noindent\textbf{3.5}
Let $(\Omega, \mathcal{F})$ and $(\chi,\mathcal{G})$ be measurable spaces, $X: \chi \to \mathbb{R}$ be a random variable and $K: \Omega\times \mathcal{G} \to [0,1]$ a prbability kernel from $(\Omega, \mathcal{F})$ to $(\chi,\mathcal{G})$. Define function $U:\Omega \to \mathbb{R}$ by $U(\omega) = \int_\chi X(x) K(\omega,dx)$ and assume that $U(\omega)$ exists for all $\omega$. Prove that $U$ is measurable. 
\begin{proof}
    When $X(x)$ is simple function, $X(x) = \sum_{i = 1}^n \alpha_i I\{A_i\}$,
    \begin{equation}
        \begin{aligned}
            U(\omega) &= \int_\chi \sum_{i = 1}^n \alpha_i I\{A_i\} K(w,dx)\\
            &= \sum_{i = 1}^n \alpha_i K(w,A_i\bigcap\chi)\ \text{is measuable}
        \end{aligned}
    \end{equation}
    When $X(x)$ is non-negative function, $X(x) = \sup \{h: h \text{ is simple and } 0\leq h \leq X\}$, 
    \begin{equation}
        \begin{aligned}
            U(\omega) &= \int_\chi \sup h(x) K(w,dx)\\
            & = \sup \int_\chi h(x) K(w,dx) \ \text{is measurable (due to dominated convergence thm)} 
        \end{aligned}
    \end{equation}
    When $X(x)$ is general function, $X = X^+ - X^-$, use the linearity of integration, we can still prove that U is measurable. 
\end{proof}



\noindent\textbf{3.6}(Limits of increasing stopping times are stopping times) Let $(\tau_n)_{\infty}$ be an almost surely increasing sequence of $\FF$-stopping times on probability space $(\Omega,\cF,\mathbb{P})$ with filtration $\FF = (\cF_n)_{n=1}^{\infty}$, which means that $\tau_n(\omega) \le \tau_{n+1}(\omega)$ for all $n\ge 1$ almost surely. Prove that $\tau(\omega)=\lim_{n\to \infty}\tau_n(\omega)$ is a $\FF$-stopping time. 

\begin{proof}
    To show $\tau$ is a stopping time, we want to show that $\forall t, \bOne{\tau \le t} \in \cF_t$. 
    \begin{align*}
        \bOne{\tau \le t} = \bOne{\lim_{n\to \infty}\tau_n \le t} = \bOne{\sup_n \tau_n \le t}   = \cap_n \bOne{\tau_n \le t} 
    \end{align*}
    Since $\forall n, \tau_n$ is a stopping time, then $\bOne{\tau_n\le t} \in \cF_t,\forall t$. And $\forall t, \cF_t$ is a $\sigma$-algebra. Thus $\forall t, \bOne{\tau \le t} = \cap_n \bOne{\tau_n \le t}  \in \cF_t$. 
\end{proof}


\noindent\textbf{3.7} (Properties of stopping times) Let $\FF = (\cF_t)_{t \in \NN}$ be a filtration, and $\tau,\tau_1,\tau_2$ be stopping times with respect to $\FF$. Show the following:


\begin{enumerate}
   \item[(a)] $\cF_\tau$ is a $\sigma$-algebra
   \item[(b)] If $\tau =k$ for some $k\ge 1$, then $\cF_\tau = \cF_k$.
   \item[(c)] If $\tau_1 \le \tau_2$, then $\cF_{\tau_1} \subseteq \cF_{\tau_2}$.
   \item[(d)] $\tau$ is $\cF_\tau$-measurable.
   \item[(e)] If $(X_t)$ is $\FF$-adapted, then $X_\tau$ is $\cF_\tau$-measurable.
   \item[(f)] $\cF_\tau$ is the smallest $\sigma$-algebra such that all $\FF$-adapted sequences $(X_t)$ satisfy $X_\tau$ is $\cF_\tau$-measurable.
\end{enumerate}

\begin{proof}

\begin{enumerate}
   \item[(a)] In this problem, we need to prove that $\cF_\tau = \set{A \in \cF_{\infty}: A\cap \set{\tau \le t} \in \cF_t, \forall t}$. According to the definition of $\sigma$-algebra, here we just need to show each requirement is satisfied. First, the whole set is $\set{\tau\le t}$, it is obvious that $\set{\tau\le t} \cap \set{\tau\le t} \in \cF_t,\forall t$. Second, $\forall A \in \cF_\tau, A^c\cap \set{\tau \le t} = \set{\tau \le t} - A\cap \set{\tau \le t}$, since both $\set{\tau \le t}$ and $A\cap \set{\tau \le t}$ are in $\cF_t,\forall t$, thus $A^c \in \cF_t, \forall t$ holds. Last, $\forall A_i \in \cF_\tau, i=1,2,..., \bracket{\bigcup_i A_i} \cap \set{\tau \le t} = \bigcup_i \bracket{ A_i \cap \set{\tau \le t}}$, since $\cF_t$ is a $\sigma$-algebra and $ A_i \cap \set{\tau \le t} \in \cF_t, \forall t$, we can conclude $\bracket{\bigcup_i A_i} \cap \set{\tau \le t} \in \cF_t, \forall t$. Above all, we have shown that $\cF_{\tau}$ is a $\sigma$-algebra on the set $\set{\tau \le t}$. 
   \item[(b)] We first prove $\cF_\tau \subseteq \cF_k$, which means $\forall A \in \cF_\tau, A \in \cF_k$. According to the definition, $\forall A \in \cF_\tau, A \cap \set{\tau \le k}\in \cF_k$, and $\set{\tau \le k}$ holds a.s. as $\tau=k$. Thus $A = A \cap \set{\tau \le k} \in \cF_k$. 

    We then prove $\cF_k \subseteq \cF_\tau$, that is $\forall B \in \cF_k, B \in \cF_\tau$. When $t \ge k$, $B \cap \set{\tau \le t} = B$, and since $B \in \cF_k \subseteq \cF_t$, we could have $B \cap \set{\tau \le t} \in \cF_t$. And for $t<k$, $B \cap \set{\tau \le t} = \emptyset \in \cF_t$. Thus we could conclude that $B \in \cF_\tau$. 

   \item[(c)] We want to prove $\forall A \in \cF_{\tau_1}$, it holds that $A \in \cF_{\tau_2}$. Recall that 
   \begin{align*}
      \cF_{\tau_1} = \set{A \in \cF_{\infty}: A\cap \set{\tau_1 \le t} \in \cF_t, \forall t}\,,\\
      \cF_{\tau_2} = \set{A \in \cF_{\infty}: A\cap \set{\tau_2 \le t} \in \cF_t, \forall t}\,.
   \end{align*}
   Since $\tau_1 \le \tau_2$, $\set{\tau_2 \le t} \subseteq \set{\tau_1\le t},\forall t$. According to the definition, we have $\forall A \in \cF_{\tau_1}$, $A \cap \set{\tau_1 \le t} \in \cF_t, \forall t$. Then $A \cap \set{\tau_2 \le t} = A\cap \set{\tau_1 \le t} - \bracket{\set{\tau_1 \le t} - \set{\tau_2 \le t}}$. And all of $A\cap \set{\tau_1 \le t}$, $\set{\tau_1 \le t}$, $\set{\tau_2 \le t}$ are in $\cF_t,\forall t$, we then have $A \cap \set{\tau_2 \le t} \in \cF_t,\forall t$. Thus there is $A \in \cF_{\tau_2}$ and further $\cF_{\tau_1} \subseteq \cF_{\tau_2}$. 




   \item[(d)] 
   \item[(e)] 
   We first clearly present the definition of $X_\tau = X_{\tau(\omega)}(\omega)$. If $\tau(\omega)=t$ for some $t$, then $X_{\tau(\omega)}(\omega) = X_t(\omega)$. If $\tau(\omega)=\infty$, then $X_{\tau(\omega)}(\omega) = \lim_{t \to \infty} X_t(\omega)$. To prove $X_\tau$ is $\cF_{\tau}$-measurable, we want to prove for any constant $a\in \RR$, $X_{\tau}^{-1}((-\infty, a)) = \set{\omega: X_{\tau}(\omega)<a} \in \cF_{\tau}$, where $\cF_{\tau} = \set{A\in \cF_{\infty}: A\cap \set{\tau \le t} \in \cF_t, \forall t}$. We first prove $\set{\omega: X_{\tau}(\omega)<a} \in \cF_{\infty}$, it can be written as 
   \begin{align*}
    \set{\omega: X_{\tau}(\omega)<a} = \set{\omega: \bigcup_{t\in \NN}\set{\tau(\omega)=t}\cap\set{X_t(\omega)<a} \bigcup \set{\tau(\omega)=\infty}\cap \set{\lim_{t\to \infty}X_t(\omega)<a }  }
   \end{align*}
   Since $\forall t$, $\set{\tau(\omega)=t}\cap\set{X_t(\omega)<a} \in \cF_{t}$, thus  $\bigcup_{t\in \NN}\set{\tau(\omega)=t}\cap\set{X_t(\omega)<a} \in \cF_{\infty}$. And $\set{\tau(\omega)=\infty} = \bigcap_{t \in \NN}\set{\tau(\omega)>t} \in \cF_{\infty}$. For the last part, 
  \begin{align*}
    \set{\lim_{t\to \infty}X_t(\omega)<a }  = \cap_{n>0}\cup_{N \in \NN} \cap_{t,s > N} \set{\abs{X_t(\omega)-X_s(\omega)}\le \frac{1}{2^n} } \bigcap \cup_{N\in \NN}\cap_{t>N}\set{X_t(\omega) < a}\,,
   \end{align*}
   where the first part represents the existence of the limitation and the second part represents the limitation is less than $a$. It is obvious that $\set{\abs{X_t(\omega)-X_s(\omega)}\le \frac{1}{2^n} } \in \cF_{\max\set{t,s}}$, and thus $\cup_{N \in \NN} \cap_{t,s > N} \set{\abs{X_t(\omega)-X_s(\omega)}\le \frac{1}{2^n} } \in \cF_{\infty}$. Similarly the second part $\cup_{N\in \NN}\cap_{t>N}\set{X_t(\omega) < a} \in \cF_{\infty}$. Above all, we have proved $\set{\omega: X_{\tau}(\omega)<a} \in \cF_{\infty}$. Next we will show $\forall t, \set{ X_{\tau}(\omega)<a} \cap \set{\tau(\omega) \le t} \in \cF_t $. 
   \begin{align}
      \set{\omega: X_{\tau}(\omega)<a} \cap \set{\tau(\omega) \le t}  = \bigcup_{s\le t} \set{\tau(\omega)=s}\cap\set{X_{s}(\omega)<a}
   \end{align}
   As $\set{\tau(\omega)=s}\cap\set{X_{s}(\omega)<a} \in \cF_{s}$, we have $\bigcup_{s\le t} \set{\tau(\omega)=s}\cap\set{X_{s}(\omega)<a} \in \cF_t, \forall t$. 

   Above all, we have proved for any constant $a\in \RR$, $X_{\tau}^{-1}((-\infty, a)) = \set{\omega: X_{\tau}(\omega)<a} \in \cF_{\tau}$



   \item[(f)]
\end{enumerate}


\end{proof}






\noindent\textbf{3.8} Prove Theorem 3.10.\\
\textbf{Theorem 3.10}
  Let $(X_t)_{t=0}^n$ be a submartingale with $X_t\ge 0$ almost surely for all $t$. Then for any $\varepsilon>0$,
  \begin{align*}
      \PP{\max_{t\in\set{0,1,...,n}}X_t \ge \varepsilon} \le \frac{\EE{X_n}}{\varepsilon}\,.
  \end{align*}

\begin{proof}
  Define event $A=\set{\max_{t\in\set{0,1,...,n}} X_t<\varepsilon }$, $\forall t=0,..,n, A_t=\set{\forall i<t: X_i<\varepsilon , X_t \ge \varepsilon}$. Note that event $A$ and $A_t,\forall t$ are exclusive. 

  According to the property of submartingale and $X_t\ge 0, \forall t$, we have
  \begin{align*}
    X_t \bOne{A_t} \le \EE{X_n \mid \cF_{t}}  \bOne{A_t} = \EE{X_n \bOne{A_t}\mid \cF_{t}}  \,,
  \end{align*}
  further according to the tower rule, we have
  \begin{align*}
   \EE{X_t \bOne{A_t}} \le \EE{\EE{X_n \bOne{A_t}\mid \cF_{t}}}  = \EE{X_n \bOne{A_t}} \,.
  \end{align*}
  Above all, there is
  \begin{align}
     \EE{X_n} &= \EE{X_n\bOne{A}}+\sum_{t=0}^n\EE{X_n\bOne{A_t}} \notag \\
      &\ge \sum_{t=0}^n\EE{X_n\bOne{A_t}}\notag\\
      &\ge \sum_{t=0}^n\EE{X_t\bOne{A_t}}\notag\\
      &\ge \varepsilon\cdot \sum_{t=0}^n\EE{\bOne{A_t}}\notag\\
      &= \varepsilon\cdot \sum_{t=0}^n \PP{A_t} \notag\\
      &= \varepsilon \cdot\PP{\bigcup_{t=0}^n A_t}\notag\\
      &= \varepsilon\cdot \PP{\max_{t\in\set{0,1,...,n}}X_t \ge \varepsilon}\notag
  \end{align}
\end{proof}












