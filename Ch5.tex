\chapter*{Chapter 5 Concentration of Measure}
\label{sec:5}

\noindent\textbf{5.1}



\begin{equation}
V(\hat{\mu })=E((\hat{\mu }-\mu)^2)
=E((\frac{1}{n}\sum_{t=1}^{n}{X_t}-\mu)^2)
=E(\frac{1}{n^2}\sum_{t=1}^{n}{(X_t - \mu)^2})
=\frac{1}{n^2}\sum_{t=1}^{n}{E(X_t - \mu)^2}
=\frac{1}{n^2}\sum_{t=1}^{n}{\sigma^2}
=\frac{\sigma^2}{n}
\end{equation}



% \noindent\textbf{5.2}
% From the definition of expectation:
% $$
% {\displaystyle \operatorname {E} (X)=\int _{-\infty }^{\infty }xf(x)\,dx}{\displaystyle \operatorname {E} (X)=\int _{-\infty }^{\infty }xf(x)\,dx}$$
% \newline

% \noindent
% However, X is a non-negative random variable thus,
% $$
% {\displaystyle \operatorname {E} (X)=\int _{-\infty }^{\infty }xf(x)\,dx=\int _{0}^{\infty }xf(x)\,dx}{\displaystyle \operatorname {E} (X)=\int _{-\infty }^{\infty }xf(x)\,dx=\int _{0}^{\infty }xf(x)\,dx}$$
% \newline

% \noindent
% From this we can derive,
% \begin{align*}
%     {\displaystyle \operatorname {E} (X)&=\int _{0}^{a}xf(x)\,dx+\int _{a}^{\infty }xf(x)\,dx\geq \int _{a}^{\infty }xf(x)\,dx\geq \int _{a}^{\infty }af(x)\,dx\\
%     &=a\int _{a}^{\infty }f(x)\,dx=a\operatorname {Pr} (X\geq a)}{\displaystyle \operatorname {E} (X)=\int _{0}^{a}xf(x)\,dx+\int _{a}^{\infty }xf(x)\,dx\\
%     &\geq \int _{a}^{\infty }xf(x)\,dx\geq \int _{a}^{\infty }af(x)\,dx\\
%  &=a\int _{a}^{\infty }f(x)\,dx=a\operatorname {Pr} (X\geq a)}
% \end{align*}

% From here, dividing through by a allows us to see that
% $$
% {\displaystyle \Pr(X\geq a)\leq \operatorname {E} (X)/a}{\displaystyle \Pr(X\geq a)\leq \operatorname {E} (X)/a}
% $$











\noindent\textbf{5.4}



(a)
\begin{equation}
P(|X|\geq \varepsilon) = P(X\geq\varepsilon)I\{X\geq 0\}+P(X\leq -\varepsilon)I\{X< 0\}
=\int_{\varepsilon}^{\infty} \frac{x}{2} exp\{ \frac{-x^2}{2}\} dx + \int_{-\infty}^{\varepsilon} \frac{-x}{2} exp\{ \frac{-x^2}{2}\} dx
\end{equation}

Calculate the above formula and get the result ,

$P(|X|\geq \varepsilon) =\frac{1}{2} exp\{ \frac{-\varepsilon^2}{2}\} + \frac{1}{2} exp\{ \frac{-\varepsilon^2}{2}\}$

$=exp\{ \frac{-\varepsilon^2}{2}\}$


(b)

Let's start with a lemma:

If X is $\sigma-$subgaussian,then $P(|X|>t) \leq exp\{ -b \varepsilon^2\}$ , where $b=exp\{ -\sigma^2\}$

The proof of lemma is omitted.

It can be seen from the first question , $P(|X|\geq \varepsilon) = exp\{ \frac{-\varepsilon^2}{2}\}$

The comparison of the two formulas shows that , $0<b\leq 1/2$ . That is, $\sigma\geq \sqrt{ln2}$

By topic condition , $\sigma = \sqrt{2-\varepsilon}$

Hence , $\varepsilon \leq 2-ln2$ , this is in contradiction with the arbitrariness of $\varepsilon$




\noindent\textbf{5.7}


(a)If X is $\sigma-$subgaussian , then $E(X)=0$,$E(X^2)\leq\sigma^2$

proof:

\begin{equation}
E(e^{\lambda X}) = \sum_{n=0}^{\infty}\frac{\lambda^n E(X^n)}{n!}=1+\lambda E(X)+\frac{\lambda^2 E(X^2)}{2}+O(\lambda^2)
\end{equation}

By definition ,

\begin{equation}
E(e^{\lambda X})\leq e^{\frac{\lambda^2 \sigma^2}{2}}=1+\frac{\lambda^2 \sigma^2}{2}+O(\lambda^2)
\end{equation}

By comparing the above two formulas and discussing the case that a approaches to 0 from above and below 0, we get the conclusion that ,

$E(X)=0$,$E(X^2)\leq\sigma^2$

(b)

If X is $\sigma-$subgaussian , then $E(X)=0$,$E(X^2)\leq\sigma^2$ .

$E(e^{c\lambda x}) = 1+\lambda E(cx)+\frac{\lambda^2 E(c^2 x^2)}{2}+O(\lambda^2)$

$\leq 1+c\lambda E(x)+\frac{\lambda^2 c^2}{2} E(x^2)+O(\lambda^2)$

$\leq 1+\frac{\lambda^2 c^2 \sigma^2}{2}+O(\lambda^2)$

$\leq e^{\frac{\lambda^2 c^2 \sigma^2}{2}}$

Hence , cX is $|c|\sigma-$subgaussian .

(c)

If $X_1$ is $\sigma_1-$subgaussian , $X_2$ is $\sigma_2-$subgaussian

then $E(X_1)=0$,$E(X_1^2)\leq\sigma_1^2$ ,$E(X_2)=0$,$E(X_2^2)\leq\sigma_2^2$

$E(e^{\lambda (x_1+x_2)}) = 1+\lambda E(x_1+x_2)+\frac{\lambda^2 E((x_1+x_2)^2)}{2}+O(\lambda^2)$

$= 1+\frac{\lambda^2}{2} Var(x_1+x_2)+O(\lambda^2)$

$= 1+\frac{\lambda^2}{2} (var(x_1)+var(x_2)+2cov(x_1,x_2))+O(\lambda^2)$

Because $x_1$, $x_2$ are independent ,

$= 1+\frac{\lambda^2}{2} (E(x_1^2) + E(x_2^2))(\lambda^2)$

$\leq 1+\frac{\lambda^2}{2} (\sigma_1^2 + \sigma_2^2)+O(\lambda^2)$

$\leq e^{\frac{\lambda^2 (\sigma_1^2 + \sigma_2^2)}{2}}$

Hence , $X_1+X_2$ is $\sqrt{\sigma_1^2 + \sigma_2^2}-$subgaussian .




\noindent\textbf{5.11}




(a)
\begin{equation}
E(e^{\lambda X}) = 1+\lambda E(X)+\frac{\lambda^2 E(X^2)}{2}+O(\lambda^2) =1+\frac{\lambda^2 E(X^2)}{2}+O(\lambda^2)
\end{equation}

If the conclusion is true, then the above formula satisfies

$\leq 1+\frac{\lambda^2}{2}(\frac{(b-a)^2}{4})+O(\lambda^2)$

So just prove:

$E(x^2)\leq (\frac{b-a}{2})^2$

$E(x^2)=var(x)=E(x-\bar{x})^2$

However,$(x-\bar{x})^2\leq(\frac{b-a}{2})^2$ . The conclusion is proved.

(b)

The proof of Hoeffding's Inequality:

Let $X_i = Z_i - E(Z_i)$ , $\bar{X} = \frac{1}{m}\sum_{i=1}^{m}X_i$

By Markov inequality , for all $\lambda >0$ , $\varepsilon > 0$,

$P(\bar{X}\geq\varepsilon) = P(e^{\lambda \bar{X}} \geq e^{\lambda\varepsilon}) \leq \frac{E(e^{\lambda \bar{X}})}{e^{\lambda\varepsilon}}$

$Z_1$,$\cdots$,$Z_m$ iid.r.v.

So,$E(e^{\lambda \bar{X}}) = \prod_{i=1}^{m} E(e^{\frac{\lambda X_i}{m}})$

By Hoeffding's lamma,

$ E(e^{\frac{\lambda X_i}{m}}) \leq e^{\frac{\lambda^2(b-a)^2}{8m^2}}$

So , $P(\bar{X}\geq\varepsilon) \leq e^{-\lambda\varepsilon}\prod_{i=1}^{m} E(e^{\frac{\lambda X_i}{m}})$

$\leq e^{-\lambda\varepsilon}e^{\frac{\lambda^2(b-a)^2}{8m}}$

$\leq e^{-\lambda\varepsilon+\frac{\lambda^2(b-a)^2}{8m}}$

Let $\lambda = \frac{4m\varepsilon}{(b-a)^2}$ , then $P(\bar{X}\geq \varepsilon) \leq e^{\frac{-2m\varepsilon^2}{(b-a)^2}}$

Similarly, we can prove the other side of the inequality.


\noindent \textbf{5.16}
By assumption $Pr(X_t\leq x)\leq x$, which means that for$\lambda <1$,
\begin{align}
\mathbb{E}\left[exp(\lambda log(\frac{1}{x_t}))\right] = \int_0^\infty P(exp(\lambda log(\frac{1}{x_t}))\geq x)dx = 1 +\int_1^\infty P(X_t \leq x^{-\frac{1}{\lambda}})dx
\end{align}
Applying the Cramer-Chernoff method,
$$P\left(\sum_{t=1}^n log(\frac{1}{X_t}) \geq \epsilon\right) = P\left(exp(\lambda \sum_{t=1}^n log(\frac{1}{X_t})) \geq exp(\lambda \epsilon) \right) \leq \left(\frac{1}{1-\lambda}\right)^n exp (-\lambda \epsilon)$$
choosing $\lambda  = \frac{\epsilon-n}{\epsilon}$ completes the claim.



\noindent \textbf{5.18(a)}
Let $\lambda >0$. Then,
$$exp(\lambda \mathbb{E}[Z]) \leq \mathbb{E}[exp(\lambda Z)] \leq \sum_{t=1}^n \mathbb{E}[exp(\lambda X_t)] \leq n exp(\frac{\lambda^2 \sigma^2}{2})$$
\noindent
Rearranging shows that,
$$\mathbb{E}(Z) \leq \frac{log(n)}{\lambda} + \frac{\lambda \sigma^2}{2}$$
\newline

\noindent
Choosing $\lambda = \frac{1}{\sigma} \sqrt{2log(n)}$ shows  that $
\mathbb{E}(Z) \leq \sqrt{2\sigma^2 log(n)}$
