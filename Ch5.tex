%!TEX root =  main.tex
\chapter*{Chapter 5 Concentration of Measure}
\label{sec:5}

\noindent\textbf{5.1}
(\textsc{Variance of average})
Let $X_{1}, X_{2}, \ldots, X_{n}$ be a sequence of independent and identically distributed random variables with mean $\mu$ and variance $\sigma^{2}<\infty$.
Let $\hat{\mu}=\frac{1}{n} \sum_{t=1}^{n} X_{t}$ and show that $\mathbb{V}[\hat{\mu}]=\mathbb{E}\left[(\hat{\mu}-\mu)^{2}\right]=\sigma^{2} / n$.

\begin{proof}
	We start straight from the definition:

	\begin{equation*}
		\begin{aligned}
		&\mathbb{V}[\hat{\mu}]\\
		= &\mathbb{E}((\hat{\mu }-\mu)^2)\\
		= &\mathbb{E}((\frac{1}{n}\sum_{t=1}^{n}{X_t}-\mu)^2)\\
		= &\mathbb{E}(\frac{1}{n^2}\sum_{t=1}^{n}{(X_t - \mu)^2})\\
		= &\frac{1}{n^2}\sum_{t=1}^{n}{\mathbb{E}(X_t - \mu)^2}\\
		= &\frac{1}{n^2}\sum_{t=1}^{n}{\sigma^2}\\
		= &\frac{\sigma^2}{n}
		\end{aligned}
	\end{equation*}			

\end{proof}

\noindent\textbf{5.2}
% From the definition of expectation:
% $$
% {\displaystyle \operatorname {E} (X)=\int _{-\infty }^{\infty }xf(x)\,dx}{\displaystyle \operatorname {E} (X)=\int _{-\infty }^{\infty }xf(x)\,dx}$$
% \newline

% \noindent
% However, X is a non-negative random variable thus,
% $$
% {\displaystyle \operatorname {E} (X)=\int _{-\infty }^{\infty }xf(x)\,dx=\int _{0}^{\infty }xf(x)\,dx}{\displaystyle \operatorname {E} (X)=\int _{-\infty }^{\infty }xf(x)\,dx=\int _{0}^{\infty }xf(x)\,dx}$$
% \newline

% \noindent
% From this we can derive,
% \begin{align*}
%     {\displaystyle \operatorname {E} (X)&=\int _{0}^{a}xf(x)\,dx+\int _{a}^{\infty }xf(x)\,dx\geq \int _{a}^{\infty }xf(x)\,dx\geq \int _{a}^{\infty }af(x)\,dx\\
%     &=a\int _{a}^{\infty }f(x)\,dx=a\operatorname {Pr} (X\geq a)}{\displaystyle \operatorname {E} (X)=\int _{0}^{a}xf(x)\,dx+\int _{a}^{\infty }xf(x)\,dx\\
%     &\geq \int _{a}^{\infty }xf(x)\,dx\geq \int _{a}^{\infty }af(x)\,dx\\
%  &=a\int _{a}^{\infty }f(x)\,dx=a\operatorname {Pr} (X\geq a)}
% \end{align*}

% From here, dividing through by a allows us to see that
% $$
% {\displaystyle \Pr(X\geq a)\leq \operatorname {E} (X)/a}{\displaystyle \Pr(X\geq a)\leq \operatorname {E} (X)/a}
% $$







\noindent\textbf{5.3}



\noindent\textbf{5.4} Let $X$ be a random variable on $\RR$ with density with respect to the Lebesgue measure of $p(x)= \abs{x}\exp\bracket{-x^2/2}/2$. Show the following:
\begin{enumerate}
	\item[(a)] $\PP{\abs{X}\ge \varepsilon} = \exp(-\varepsilon^2/2)$
	\item[(b)] $X$ is not $\sqrt{2-\varepsilon}$-subgaussian for any $\varepsilon>0$.
\end{enumerate}

\begin{proof}
\begin{enumerate}
	\item[(a)] 
	\begin{align*}
		\PP{\abs{X}\ge \varepsilon} &=\int^{-\varepsilon}_{-\infty} 
		\frac{\abs{x}}{2}\exp\bracket{-x^2/2} d x +  \int_{\varepsilon}^{+\infty} \frac{\abs{x}}{2} \exp\bracket{-x^2/2} d x \\
		&=\int^{-\varepsilon}_{-\infty} -\frac{x}{2} \exp\bracket{-x^2/2}/2 d x +  \int_{\varepsilon}^\infty \frac{x}{2} \exp\bracket{-x^2/2} d x \\
		&= \int_{\varepsilon}^\infty x \exp\bracket{-x^2/2} d x \\
		&= \exp\bracket{-\varepsilon^2/2}\,.
	\end{align*}
	\item[(b)] 
	According to the definition of the sub-gaussian random variables, to prove $X$ is not $\sqrt{2-\varepsilon}$-subgaussian, we want to prove $\exists \lambda \in \RR$ such that 
	\begin{align*}
		\EE{\exp\bracket{\lambda X}} \le \exp\bracket{\frac{\lambda^2 (2-\varepsilon)}{2}} \triangleq g(\lambda) \,.
	\end{align*}
	For the LHS, there is
	\begin{align*}
		\EE{\exp\bracket{\lambda X}} &= 	\int_{-\infty}^{+\infty} \exp(\lambda x) \frac{|x|}{2} \exp(-x^2/2) dx \\
		&= \frac{1}{2}\exp(\lambda^2/2)\int_{-\infty}^{+\infty} \abs{t+\lambda} \exp(-t^2/2)dt \\
		&= \frac{\exp(\frac{\lambda^2}{2})}{2} \bracket{\int_{-\infty}^{-\lambda} (-t-\lambda) \exp(-t^2/2)dt+\int_{-\lambda}^{+\lambda} (t+\lambda) \exp(-t^2/2)dt + \int_{\lambda}^{+\infty}(t+\lambda) \exp(-t^2/2)dt} \\
		&= \frac{\exp(\frac{\lambda^2}{2})}{2} \bracket{{2\exp(-\frac{\lambda^2}{2})} + 2\lambda \int_0^\lambda \exp(-t^2/2)dt } \\
		&= 1 + \lambda \exp(\frac{\lambda^2}{2})\int_{0}^{\lambda} \exp(-t^2/2) dt \triangleq f(\lambda) \,,.
	\end{align*}
	Define $F(\lambda) = f(\lambda) - g(\lambda)$, now we want to prove $\exists \lambda \in \RR, F(\lambda)>0$. By computation, we found that $F(0)=0, F'(0) = 0, F''(0)=\varepsilon >0$. We can then conclude that $(0,0)$ is a minima of $F$, thus $\exists \delta>0, \forall \lambda\in (-\delta,\delta), F(\lambda)>0$. We then get the desired result. 
\end{enumerate}
\end{proof}



% (a)
% \begin{equation}
% P(|X|\geq \varepsilon) = P(X\geq\varepsilon)I\{X\geq 0\}+P(X\leq -\varepsilon)I\{X< 0\}
% =\int_{\varepsilon}^{\infty} \frac{x}{2} exp\{ \frac{-x^2}{2}\} dx + \int_{-\infty}^{\varepsilon} \frac{-x}{2} exp\{ \frac{-x^2}{2}\} dx
% \end{equation}

% Calculate the above formula and get the result ,

% $P(|X|\geq \varepsilon) =\frac{1}{2} exp\{ \frac{-\varepsilon^2}{2}\} + \frac{1}{2} exp\{ \frac{-\varepsilon^2}{2}\}$

% $=exp\{ \frac{-\varepsilon^2}{2}\}$


% (b)

% Let's start with a lemma:

% If X is $\sigma-$subgaussian,then $P(|X|>t) \leq exp\{ -b \varepsilon^2\}$ , where $b=exp\{ -\sigma^2\}$

% The proof of lemma is omitted.

% It can be seen from the first question , $P(|X|\geq \varepsilon) = exp\{ \frac{-\varepsilon^2}{2}\}$

% The comparison of the two formulas shows that , $0<b\leq 1/2$ . That is, $\sigma\geq \sqrt{ln2}$

% By topic condition , $\sigma = \sqrt{2-\varepsilon}$

% Hence , $\varepsilon \leq 2-ln2$ , this is in contradiction with the arbitrariness of $\varepsilon$

\noindent\textbf{5.5}



\noindent\textbf{5.6}


\noindent\textbf{5.7}


(a)If X is $\sigma-$subgaussian , then $E(X)=0$,$E(X^2)\leq\sigma^2$

proof:

\begin{equation}
E(e^{\lambda X}) = \sum_{n=0}^{\infty}\frac{\lambda^n E(X^n)}{n!}=1+\lambda E(X)+\frac{\lambda^2 E(X^2)}{2}+O(\lambda^2)
\end{equation}

By definition ,

\begin{equation}
E(e^{\lambda X})\leq e^{\frac{\lambda^2 \sigma^2}{2}}=1+\frac{\lambda^2 \sigma^2}{2}+O(\lambda^2)
\end{equation}

By comparing the above two formulas and discussing the case that a approaches to 0 from above and below 0, we get the conclusion that ,

$E(X)=0$,$E(X^2)\leq\sigma^2$

(b)

If X is $\sigma-$subgaussian , then $E(X)=0$,$E(X^2)\leq\sigma^2$ .

$E(e^{c\lambda x}) = 1+\lambda E(cx)+\frac{\lambda^2 E(c^2 x^2)}{2}+O(\lambda^2)$

$\leq 1+c\lambda E(x)+\frac{\lambda^2 c^2}{2} E(x^2)+O(\lambda^2)$

$\leq 1+\frac{\lambda^2 c^2 \sigma^2}{2}+O(\lambda^2)$

$\leq e^{\frac{\lambda^2 c^2 \sigma^2}{2}}$

Hence , cX is $|c|\sigma-$subgaussian .

(c)

If $X_1$ is $\sigma_1-$subgaussian , $X_2$ is $\sigma_2-$subgaussian

then $E(X_1)=0$,$E(X_1^2)\leq\sigma_1^2$ ,$E(X_2)=0$,$E(X_2^2)\leq\sigma_2^2$

$E(e^{\lambda (x_1+x_2)}) = 1+\lambda E(x_1+x_2)+\frac{\lambda^2 E((x_1+x_2)^2)}{2}+O(\lambda^2)$

$= 1+\frac{\lambda^2}{2} Var(x_1+x_2)+O(\lambda^2)$

$= 1+\frac{\lambda^2}{2} (var(x_1)+var(x_2)+2cov(x_1,x_2))+O(\lambda^2)$

Because $x_1$, $x_2$ are independent ,

$= 1+\frac{\lambda^2}{2} (E(x_1^2) + E(x_2^2))(\lambda^2)$

$\leq 1+\frac{\lambda^2}{2} (\sigma_1^2 + \sigma_2^2)+O(\lambda^2)$

$\leq e^{\frac{\lambda^2 (\sigma_1^2 + \sigma_2^2)}{2}}$

Hence , $X_1+X_2$ is $\sqrt{\sigma_1^2 + \sigma_2^2}-$subgaussian .



\noindent\textbf{5.8}
(\textsc{Properties of subgaussian random variables (ii)})
Let $X_{i}$ be $\sigma_i$-subgaussian for $i \in\{1,2\}$ with $\sigma_i \geq 0$.
Prove that $X_{1}+X_{2}$ is $(\sigma_1 + \sigma_2)$-subgaussian.
Do \textit{not} assume independence of $X_1$ and $X_2$.

\begin{proof}
	We start straight from the definition:

	\begin{equation*}
	\begin{aligned}
	&\mathbb{E}[\exp(\lambda(X_{1}+X_{2}))]\\
	\leq &\mathbb{E}[\exp(\lambda p X_{1})]^\frac{1}{p} \mathbb{E}[\exp(\lambda q X_{2})]^\frac{1}{q}\\
	\leq &\exp(\lambda^2 p^2 \sigma_1^2 / 2)^\frac{1}{p} \exp(\lambda^2 q^2 \sigma_2^2 / 2)^\frac{1}{q}\\
	= &\exp(\frac{\lambda^2(p \sigma_1^2 + q \sigma_2^2)}{2})\\
	= &\exp(\frac{\lambda^2(\sigma_1^2 + \sigma_2^2)}{2}),
	\end{aligned}
	\end{equation*}
	where the first inequality holds according to Hölder's inequality and the last equality holds with $p = \frac{\sigma_1^2 + \sigma_2^2}{\sigma_2^2}$.
\end{proof}

\noindent\textbf{5.9}
(Properties of moment/cumulative-generating functions) Let $X$ be a real-valued random variable and let $M_X(\lambda) = \EE{\exp(\lambda X)}$ be its moment-generating function defined over $\text{dom}(M_X) \subseteq \RR$, where the expectation takes on finite values. Show that the following properties hold:
\begin{enumerate}
	\item[(a)] $M_X$ is convex, and in particular $\text{dom}(M_X)$ is an interval containing zero.
	\item[(b)] $M_X(\lambda) \ge e^{\lambda \EE{X}}$ for all $\lambda \in \text{dom}(M_X)$.
	\item[(c)] For any $\lambda$ in the interior of $\text{dom}(M_X)$, $M_X$ is infinitely many times differentiable.
	\item[(d)] Let $M^{(k)}_X(\lambda) = \frac{d^k}{d\lambda^k}M_X(\lambda)$. Then, for $\lambda$ in the interior of $\text{dom}(M_X)$, $M^{(k)(\lambda)} = \EE{X^k \exp(\lambda X)}$. 
	\item[(e)] Assuming $0$ is in the interior of $\text{dom}(M_X)$, $M^{(k)(0)} = \EE{X^k}$ (hence the name of $M_X$).
	\item[(f)] $\psi_X$ is convex (that is, $M_X$ is log-convex). 
\end{enumerate}

\begin{proof}

\begin{enumerate}
	\item[(a)] To prove $M_X$ is convex, we want to prove $\forall \alpha \in (0,1), a,b\in \text{dom}(M_X)$, there is $M_X(\alpha 
	 a + (1-\alpha)b) \le \alpha M_X(a) + (1-\alpha)M_X(b)$. 
	 \begin{align*}
	 	M_X(\alpha a + (1-\alpha)b) &= \EE{ \exp\bracket{ \alpha a + (1-\alpha)b X }} \\
	 	&\le \EE{ \alpha \exp\bracket{ a X}+ (1-\alpha)\exp\bracket{b X} } \\
	 	&= \alpha \EE{\exp\bracket{ a X}}+(1-\alpha)\EE{\exp\bracket{b X}} = \alpha M_X(a) + (1-\alpha)M_X(b)\,,
	 \end{align*}
	 where the inequality comes from the convexity of $x \to \exp(x)$.

	 To prove $\text{dom}(M_X)$ is an interval containing zero, we want to prove $M_X(0)<\infty$. It is obvious that $ M_X(0) = \EE{\exp(0\cdot X)} = \EE{\exp(0)} =1<\infty $. 
	\item[(b)] For all $\lambda \in \text{dom}M_X$, we have
	\begin{align*}
		M_X(\lambda) = \EE{ \exp(\lambda X) } \ge \exp\bracket{\EE{\lambda X}} = \exp\bracket{\lambda\EE{ X}}\,,
	\end{align*}
	where the inequality comes from the convexity of $x \to \exp(x)$.
	\item[(c)] TBD
	\item[(d)] $M^{(1)}_X(\lambda) = \frac{d}{d\lambda} \EE{\exp(\lambda X)}= \EE{\frac{d}{d\lambda} \exp(\lambda X)} = \EE{X \exp(\lambda X)}$. 
	Recursively, we have $M^{(k)}_X(\lambda) = \EE{X^k \exp(\lambda X)}$. 
	\item[(e)] According to the result of (d), we have $M^{(k)}_X(0) = \EE{X^k \exp(0\cdot X)} = \EE{X^k}$.
	\item[(f)] To prove $\psi_X$ is convex, we want to prove $\forall \alpha \in (0,1), a,b\in \text{dom}(\psi_X)$, there is $\psi_X(\alpha a + (1-\alpha)b) \le \alpha \psi_X(a) + (1-\alpha)\psi_X(b)$.
	\begin{align*}
		\psi_X(\alpha a + (1-\alpha)b) &= \log M_X( \alpha a + (1-\alpha)b ) \\
		&= \log \EE{\exp\bracket{ \alpha a + (1-\alpha)b}X} \\
		&= \log \EE{\exp\bracket{ \alpha aX } \exp((1-\alpha)bX)} \\
		&= \log \EE{\bracket{\exp\bracket{ aX }}^{\alpha} \bracket{\exp(bX)}^{(1-\alpha)}}\\
		&\le  \log \bracket{ \EE{\exp\bracket{ aX }}^{\alpha } \EE{\exp(bX)}^{(1-\alpha)} } \\
		&= \alpha \log \EE{ \exp\bracket{ aX } } + (1-\alpha) \EE{\exp(bX)} = \alpha \psi_X(a) + (1-\alpha)\psi_X(b) \,,
	\end{align*}
	where the inequality comes from the Hölder's inequality. 
\end{enumerate}

\end{proof}


\noindent\textbf{5.10}(Large deviation theory) Let $X,X_1,X_2,...,X_n$ be a sequence of independent and identically distributed random variables with zero mean and moment-generating function $M_X$ with $dom(M_X)=\RR.$ Let$\hat{\mu}_n=\frac{1}{n}\sum_{t=1}^nX_t$.
\begin{enumerate}
    \item[(a)]Show that for any $\epsilon>0$,
        \begin{align}
            \frac{1}{n}\log\PP{\hat{\mu}_n\geq\epsilon}\leq -\psi_X^{\ast}(\epsilon)=-\underset{\lambda}{\sup}(\lambda\epsilon-\log M_X(\lambda))
        \end{align}
    \item[(b)]Show that when X is a Rademacher variable$(\PP{X=1}=\PP{X=-1}=\frac{1}{2})$,$\psi_X^{\ast}(\epsilon)=\frac{1+\epsilon}{2}\log(1+\epsilon)+\frac{1-\epsilon}{2}\log(1-\epsilon)$when$\abs{\epsilon}\le 1$ and $\psi_X^{\ast}(\epsilon)=+\infty$,otherwise.
\end{enumerate}

\begin{proof}
\begin{enumerate}
	\item[(a)] For all $\lambda\in domM_X$,we have 
	    \begin{align*}
	        \frac{1}{n}\log\PP{\hat{\mu}_n\geq\epsilon}&=\frac{1}{n}\log\PP{\exp{(\lambda n\hat{\mu}_n)}\geq\exp{(\lambda n \epsilon)}}\\
	        &\leq \frac{1}{n}\log(\EE{\exp{(\lambda n\hat{\mu}_n)}}\exp{(-\lambda n \epsilon)})\\
	        &=\frac{1}{n}\log(\prod_{t=1}^n\EE{\exp{(\lambda X_t)}}\exp{(-\lambda n \epsilon)})\\
	        &=\frac{1}{n}\log(\EE{\exp{(\lambda X)}}^n\exp{(-\lambda n \epsilon)})\\
	        &=\log(\EE{\exp{(\lambda X)}}\exp{(-\lambda\epsilon)})\\
	        &=-(\lambda\epsilon-\log M_X(\lambda))
	    \end{align*}
        Since it holds for all $\lambda\in domM_X$,we can imply
        \begin{align*}
            \frac{1}{n}\log\PP{\hat{\mu}_n\geq\epsilon}\leq -\psi_X^{\ast}(\epsilon)=-\underset{\lambda}{\sup}(\lambda\epsilon-\log M_X(\lambda))
        \end{align*}

	\item[(b)]
\end{enumerate}
\end{proof}


\noindent\textbf{5.11} (Hoeffding's lemma) Suppose that $X$ is zero mean and $X \in [a,b]$ almost surely for constants $a <b$.
\begin{enumerate}
	\item[(a)]Show that $X$ is $(b-a)/2$-subgaussian.
	\item[(b)]Prove Hoeffding's inequality (Lemma \ref{lem:hoeffding}).
\end{enumerate}
\begin{lemma}[Hoeffding's inequality]\label{lem:hoeffding}
	For a zero-mean random variable $X$ such that $X \in [a,b]$ almost surely for real values $a <b$, then $M_X(\lambda) \le \exp(\lambda^2 (b-a)^2 / 8)$. Applying the Cramér–Chernoff method shows that if $X_1,X_2,\ldots,X_n$ are independent and $X_t \in [a_t,b_t]$ almost surely with $a_t < b_t$ for all $t$. Then, 
	\begin{align}
		\PP{\frac{1}{n} \sum_{t=1}^n \bracket{X_t - \EE{X_t}} \ge \varepsilon } \le \exp\bracket{ \frac{-2n^2\varepsilon^2}{\sum_{t=1}^n (b_t-a_t)^2 } }\,.
	\end{align}
\end{lemma}

\begin{proof}
\begin{enumerate}
	\item[(a)]To show $X$ is $(b-a)/2$-subgaussian, we want to prove $\forall \lambda \in \RR, \EE{\exp(\lambda X)} \le \exp\bracket{ \frac{\lambda^2 (b-a^2)}{8} }$.

	According to the convexity of $x \to \exp(x)$ and Jensen's inequality, we have $\forall X \in [a,b]$,
	\begin{align*}
		\exp(\lambda X) = \exp\bracket{ \frac{b-X}{b-a}\lambda a + \frac{X-a}{b-a}\lambda b } \le \frac{b-X}{b-a}\exp(\lambda a)+ \frac{X-a}{b-a}\exp(\lambda b)\,.
	\end{align*}
	Thus, 
	\begin{align*}
		\EE{\exp(\lambda X)} &\le \frac{b-\EE{X}}{b-a}\exp(\lambda a)+ \frac{\EE{X}-a}{b-a}\exp(\lambda b) \\
		&= \frac{b}{b-a}\exp(\lambda a)+ \frac{-a}{b-a}\exp(\lambda b)\\
		&= (1-\theta)\exp(\lambda a)+\theta \exp(\lambda b) ~~~\bracket{\text{By letting $\theta = \frac{-a}{b-a}$}} \\
		&= \exp(\lambda a) \bracket{1-\theta+\theta \exp(\lambda b - \lambda a)} \\
		&= \exp(-\lambda(b-a)\theta) \bracket{ 1-\theta+\theta \exp(\lambda b - \lambda a) } ~~~\bracket{\text{By representing $a$ using $\theta$}}\\
		&= \exp\bracket{-\theta u + \log(1-\theta+\theta\exp(u))} ~~~\bracket{\text{By letting $u=\lambda(b-a)$} } \\
		&= \exp(\phi(u)), ~~~\text{where}~~ \phi(u)=-\theta u + \log(1-\theta+\theta\exp(u))\,.
	\end{align*}
	We next want to find the upper bound for $\exp(\phi(u))$. According to the Taylor's theorem with mean-values forms of the remainder, we have 
	\begin{align*}
		\phi(u) &= \phi(0) + u\phi'(0) + \frac{1}{2}u^2 \phi''(v),~~~~\text{where}~v \in (0,u)\\
		&= \frac{1}{2}u^2 \phi''(v) ~~~\bracket{\text{Since $\phi(0) = \phi'(0) = 0$}}\\
		&= \frac{1}{2}u^2 \cdot \frac{\theta \exp(v)}{1-\theta+\theta \exp(v)}\bracket{1-\frac{\theta \exp(v)}{1-\theta+\theta \exp(v)}} \\
		&\le \frac{1}{2}u^2 \cdot \frac{1}{4} =\frac{1}{8}u^2 = \frac{\lambda^2 (b-a)^2}{8}\,.
	\end{align*}
	Above all, we have proved that $\forall \lambda \in \RR, \EE{\exp(\lambda X)} \le \exp(\phi(u)) \le \exp\bracket{\frac{\lambda^2 (b-a)^2}{8}}$. 


	\item[(b)]  We only give the upper tail of the Hoeffding's Inequality using the Hoeffding's Lemma since the lower tail has a similar proof. Applying the Hoeffding' Lemma and the Chernoff bound technique immediately
	shows that
	\begin{align*}
		\mathbb{P}\left(\frac{1}{n}\sum^n_{t=1}(X_t -\mathbb{E}[X_t])\geq\epsilon\right)&=\mathbb{P}\left(\sum^n_{t=1}(X_t-\mathbb{E}[X_t])\geq n\epsilon \right)\\
		&\leq\mathbb{E}\left[\exp\left(\lambda\sum^n_{t=1}(X_t-\mathbb{E}[X_t]) \right) \right]e^{-\lambda n\epsilon}\\
		&=\left(\Pi^n_{t=1}\mathbb{E}[\exp\left(\lambda(X_t-\mathbb{E}[X_t])\right)] \right)e^{-\lambda n\epsilon}\\
		&\leq\left(\Pi^n_{t=1}e^{\frac{\lambda^2(b_t-a_t)^2}{8}} \right)e^{-\lambda n\epsilon}\,,
	\end{align*}
	where $\lambda\geq0$. Minimizing the RHS of the above inequality over $\lambda$ shows that
	\begin{align*}
		\mathbb{P}\left(\frac{1}{n}\sum^n_{t=1}(X_t -\mathbb{E}[X_t])\geq\epsilon\right)\leq \min_{\lambda\geq0}\left(\Pi^n_{t=1}e^{\frac{\lambda^2(b_t-a_t)^2}{8}} \right)e^{-\lambda n\epsilon} =\exp\left(\frac{-2n^2\epsilon^2}{\sum^n_{t=1}(b_t-a_t)^2}\right)\,.
	\end{align*}
\end{enumerate}
\end{proof}



% (a)
% \begin{equation}
% E(e^{\lambda X}) = 1+\lambda E(X)+\frac{\lambda^2 E(X^2)}{2}+O(\lambda^2) =1+\frac{\lambda^2 E(X^2)}{2}+O(\lambda^2)
% \end{equation}

% If the conclusion is true, then the above formula satisfies

% $\leq 1+\frac{\lambda^2}{2}(\frac{(b-a)^2}{4})+O(\lambda^2)$

% So just prove:

% $E(x^2)\leq (\frac{b-a}{2})^2$

% $E(x^2)=var(x)=E(x-\bar{x})^2$

% However,$(x-\bar{x})^2\leq(\frac{b-a}{2})^2$ . The conclusion is proved.

% (b)

% The proof of Hoeffding's Inequality:

% Let $X_i = Z_i - E(Z_i)$ , $\bar{X} = \frac{1}{m}\sum_{i=1}^{m}X_i$

% By Markov inequality , for all $\lambda >0$ , $\varepsilon > 0$,

% $P(\bar{X}\geq\varepsilon) = P(e^{\lambda \bar{X}} \geq e^{\lambda\varepsilon}) \leq \frac{E(e^{\lambda \bar{X}})}{e^{\lambda\varepsilon}}$

% $Z_1$,$\cdots$,$Z_m$ iid.r.v.

% So,$E(e^{\lambda \bar{X}}) = \prod_{i=1}^{m} E(e^{\frac{\lambda X_i}{m}})$

% By Hoeffding's lamma,

% $ E(e^{\frac{\lambda X_i}{m}}) \leq e^{\frac{\lambda^2(b-a)^2}{8m^2}}$

% So , $P(\bar{X}\geq\varepsilon) \leq e^{-\lambda\varepsilon}\prod_{i=1}^{m} E(e^{\frac{\lambda X_i}{m}})$

% $\leq e^{-\lambda\varepsilon}e^{\frac{\lambda^2(b-a)^2}{8m}}$

% $\leq e^{-\lambda\varepsilon+\frac{\lambda^2(b-a)^2}{8m}}$

% Let $\lambda = \frac{4m\varepsilon}{(b-a)^2}$ , then $P(\bar{X}\geq \varepsilon) \leq e^{\frac{-2m\varepsilon^2}{(b-a)^2}}$

% Similarly, we can prove the other side of the inequality.


\noindent \textbf{5.16}
By assumption $Pr(X_t\leq x)\leq x$, which means that for$\lambda <1$,
\begin{align}
\mathbb{E}\left[exp(\lambda log(\frac{1}{x_t}))\right] = \int_0^\infty P(exp(\lambda log(\frac{1}{x_t}))\geq x)dx = 1 +\int_1^\infty P(X_t \leq x^{-\frac{1}{\lambda}})dx
\end{align}
Applying the Cramer-Chernoff method,
$$P\left(\sum_{t=1}^n log(\frac{1}{X_t}) \geq \epsilon\right) = P\left(exp(\lambda \sum_{t=1}^n log(\frac{1}{X_t})) \geq exp(\lambda \epsilon) \right) \leq \left(\frac{1}{1-\lambda}\right)^n exp (-\lambda \epsilon)$$
choosing $\lambda  = \frac{\epsilon-n}{\epsilon}$ completes the claim.



\noindent \textbf{5.18} (\textsc{Expectation of maximum})
Let $X_{1}, \ldots, X_{n}$ be a sequence of $\sigma$-subgaussian random variables
(possibly dependent) and $Z=\max _{t \in[n]} X_{t}$. Prove that

\begin{itemize}
	\item[(a)] $\mathbb{E}[Z] \leq \sqrt{2 \sigma^{2} \log (n)}$.
	\item[(b)] $\mathbb{P}\left(Z \geq \sqrt{2 \sigma^{2} \log (n / \delta)}\right) \leq \delta \text { for any } \delta \in(0,1)$.
\end{itemize}

\begin{proof}
	\begin{itemize}
		\item[(a)] Let $\lambda >0$.
		Then,
		\begin{equation*}
			\exp(\lambda \mathbb{E}[Z]) \leq \mathbb{E}[\exp(\lambda Z)] \leq \sum_{t=1}^n \mathbb{E}[\exp(\lambda X_t)] \leq n \exp(\frac{\lambda^2 \sigma^2}{2}).
		\end{equation*}
		
		Rearranging shows that
		\begin{equation*}
			\mathbb{E}(Z) \leq \frac{log(n)}{\lambda} + \frac{\lambda \sigma^2}{2}.
		\end{equation*}
	
		Choosing $\lambda = \frac{1}{\sigma} \sqrt{2log(n)}$ shows that $\mathbb{E}(Z) \leq \sqrt{2\sigma^2 log(n)}$
		
		\item[(b)] First notice that
		\begin{equation*}
			\begin{aligned}
				\mathbb{P}\left(Z \geq \sqrt{2 \sigma^{2} \log (n / \delta)}\right)
				&= \mathbb{P}\left(\exists i: X_i \geq \sqrt{2 \sigma^{2} \log (n / \delta)}\right)\\
				&\leq \sum_{i=1}^n \mathbb{P}\left(X_i \geq \sqrt{2 \sigma^{2} \log (n / \delta)}\right),
			\end{aligned}
		\end{equation*} 
		which is given directly by a union bound.

		Then, according to Theorem 5.3, we have $\mathbb{P}\left(X_i \geq \sqrt{2 \sigma^{2} \log (n / \delta)}\right)
		\leq \frac{\delta}{n}$ to complete the proof.
	\end{itemize}
\end{proof}
