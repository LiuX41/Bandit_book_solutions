%!TEX root =  main.tex


\chapter*{Chapter 11 The Exp3 Algorithm}
\label{sec:11}

\noindent\textbf{11.2} Let $\pi$ be a deterministic policy, and we define $x_{ti}=0$ if $A_t=i$ otherwise $x_{ti}=1$. The deterministic policy collects zero rewards all time,
\begin{equation}
    \max _{i \in[k]} \sum_{t=1}^{n} x_{t i} \geq \frac{1}{k} \sum_{t=1}^{n} \sum_{i=1}^{k} x_{t i}=\frac{n(k-1)}{k} \notag
\end{equation}

\noindent\textbf{11.5} Let $P$ be a probability vector with nonzero components and let $A\sim P$. Suppose $\hat{X}$ is a function such that for all $x\in \mathbb{R}^k$,
\begin{equation}
    \mathbb{E}\left[\hat{X}\left(A, x_{A}\right)\right]=\sum_{i=1}^{k} P_{i} \hat{X}\left(i, x_{i}\right)=x_{1}\notag
\end{equation}
Show that there exists an $a\in \mathbb{R}^k$ such that $<a,P>=0$ and for all $i$ and $z$ in their respective domains, $\hat{X}(i, z)=a_{i}+\frac{\mathbb{I}\{i=1\} z}{P_{1}}$
\begin{proof}
    Let $x, x'$ be arbitrary but agree on the first component $x_1=x_1'$. Let $f(x)=\sum_{i=1}^{k} P_{i} \hat{X}\left(i, x_{i}\right)$ Note that,
    \begin{equation}
        0=f(x)-f(x')=\sum_{i=j}^{k} P_{j} \hat{X}\left(j, x_{j}\right)\notag
    \end{equation}
    for all $j>1$. Since $x, x'$ are arbitrary, $\hat{X}(j, )=const$. Let $a_j$ equal to $\hat{X}(j, )$.
    \par Further, let $a_1=\hat{X}(1, 0)$ and then given any $x_1\in \mathbb{R}$, $\hat{X}\left(1, x_{1}\right)=a_{1}+x_{1} / P_{1}$.
    \par Finally, let $x$ be such that $x_1=0$. Then $0=f(x)=\sum_{i} P_{i} a_{i}$.
\end{proof}

\noindent\textbf{11.7} First, note that if $G=-\log (-\log (U))$ then $\mathbb{P}(G \leq g)=e^{-\exp (-g)}$.
\begin{equation}
    \begin{aligned}
        \mathbb{P}\left(\log a_{i}+G_{i} \geq \max _{j \in[k]} \log a_{j}+G_{j}\right) &=\mathbb{E}\left[\prod_{j \neq i} \mathbb{P}\left(\log a_{j}+G_{j} \leq \log a_{i}+G_{i} \mid G_{i}\right)\right] \notag\\
        &=\mathbb{E}\left[\prod_{j \neq i} \exp \left(-\frac{a_{j}}{a_{i}} \exp \left(-G_{i}\right)\right)\right] \notag\\
        &=\mathbb{E}\left[U_{i}^{\sum_{j \neq i} \frac{a_{j}}{a_{i}}}\right] \notag\\
        &=\frac{1}{1+\sum_{j \neq i} \frac{a_{j}}{a_{i}}} \notag\\
        &=\frac{a_{i}}{\sum_{j=1}^{k} a_{j}}\notag
        \end{aligned}
\end{equation}

\noindent\textbf{11.8} (\textsc{Exp3 as follow-the-perturbed-leader}) Let $(Z_{ti})_{ti}$ be a collection of independent and identically distributed random variables. The follow-the-perturbed-leader (FTPL) algorithm chooses
\begin{align*}
A_{t}=\argmax_{i \in[k]}\bracket{Z_{ti}-\eta \sum_{s=1}^{t-1} \hat{Y}_{s i}}\,.
\end{align*}
Show that if $Z_{ti}$ is a standard Gumbel, then follow-the-perturbed-leader is the same as Exp3. 
\begin{proof}
% Let $\hat{X}_{ti}=1-\hat{Y}_{ti}$. 
Recall in Exp3, 
\begin{align*}
    \PP{A_t=i} = \frac{\exp\bracket{\eta \sum_{s=1}^{t-1} \hat{X}_{si} }}{\sum_{j=1}^K \exp\bracket{\eta \sum_{s=1}^{t-1} \hat{X}_{sj} }} \,, \text{ where } \hat{X}_{ti} = 1-\frac{ \bOne{A_t=i}(1-X_t) }{\PP{A_t=i} } = 1-Y_{ti}\,. 
 \end{align*}
 And in FTPL, 
 \begin{align*}
\PP{A_t=i} &= \PP{ i =\argmax_{j\in[K]} Z_{tj}-\eta \sum_{s=1}^{t-1} \hat{Y}_{sj}} \\
&= \PP{ i = \argmax_{j\in[K]}Z_{tj}-\eta \sum_{s=1}^{t-1} (1-\hat{X}_{sj})  } \\
&= \PP{ i = \argmax_{j\in[K]}Z_{tj}- \eta(t-1)+ \eta \sum_{s=1}^{t-1} \hat{X}_{sj}  } \\
&= \PP{ i = \argmax_{j\in[K]}Z_{tj}+ \eta \sum_{s=1}^{t-1} \hat{X}_{sj}  } \\
&= \PP{ i = \argmax_{j\in[K]}Z_{tj}+ \log a_j  }  \\
&= \frac{a_i}{\sum_{j=1}^K a_j} = \frac{\exp\bracket{\eta \sum_{s=1}^{t-1} \hat{X}_{si} }}{\sum_{j=1}^K \exp\bracket{\eta \sum_{s=1}^{t-1} \hat{X}_{sj} }} \,,
 \end{align*}
 where we set $a_i = \exp\bracket{\eta \sum_{s=1}^{t-1}\hat{X}_{si}}$ and apply the result of 11.7. 

 Above all, we have shown that the policy of FTPL is equivalent to that of Exp3. 


\end{proof}

